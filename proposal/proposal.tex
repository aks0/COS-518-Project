% Use this template to write your solutions

\documentclass[10pt]{article}
% \documentclass[twocolumn]{sigplanconf}

\usepackage{fullpage}
\usepackage{color}
\usepackage{graphicx}
\usepackage{epsfig}
\usepackage{amsthm}
\usepackage{latexsym}
\usepackage{amssymb}
\usepackage{amsmath}

\newcommand{\newfontobj}[2]{
  \newcommand{#1}[1]{
    \expandafter\def\csname##1\endcsname{{#2 ##1}}}}

\newfontobj{\class}{\rm} % Typeset Classes in roman font

% Some standard classes (use in only mathmode)
% Usage example: $\P \subseteq \NP$ and we believe that $\NP$ is not 
%  equal to $\P$.


\class{PSPACE}	
\class{L}
\class{BPL}
\class{RL}
\class{NC}
\class{ZPL}
\class{NPSPACE}	
\class{ASPACE}	
\class{NL}
\class{EXP}
\class{NEXP}
\class{coNEXP}
\class{NE}
\class{E}
\class{AM}		
\class{MA}
\class{NP}
\class{DNP}
\class{UP}
\class{P}
\class{RP}
\class{BPP}
\class{ZPP}
\class{EXPSPACE}
\class{coNP}
\class{coRP}
\class{coAM}
\class{PH}
\class{IP}
\class{PCP}
\class{MIP}

% operator classes.
\class{BP}

% these commands should be used in math mode - $ $
\newcommand{\SHARPP}{{\#\rm{P}}}
\newcommand{\PARITYP}{{\oplus\rm{P}}}

% math operators...
\DeclareMathOperator{\poly}{poly}
\DeclareMathOperator{\Majority}{Majority}
\DeclareMathOperator{\quasipoly}{quasi-poly}
\DeclareMathOperator{\polylog}{poly-log}
\DeclareMathOperator{\superpoly}{super-poly}
\DeclareMathOperator{\DTISP}{DTISP}
\DeclareMathOperator{\DSPACE}{DSPACE}
\DeclareMathOperator{\DTIME}{DTIME}
\DeclareMathOperator{\NSPACE}{NSPACE}
\DeclareMathOperator{\NTIME}{NTIME}
\DeclareMathOperator{\BPTIME}{BPTIME}
\DeclareMathOperator{\RTIME}{RTIME}
\DeclareMathOperator{\ZPTIME}{ZPTIME}
\DeclareMathOperator{\BPSPACE}{BPSPACE}
\DeclareMathOperator{\RSPACE}{RSPACE}
\DeclareMathOperator{\ZPSPACE}{ZPSPACE}


% Complexity class
\newcommand{\CC}{\mathcal{C}}


% \lecture{number}{date}{title}{scribe}
\newcommand{\lecture}[4]{
\noindent
\fbox{
\begin{minipage}{6.2in}
  {\bf CS 710: Complexity Theory} \hfill #2
  \begin{center}
    {\Large Lecture #1: #3} \\[3mm]
  \end{center}
Instructor: Dieter van Melkebeek \hfill Scribe: #4
\end{minipage}
}
\bigskip

\bigskip
}
% \homework{number}{date}{name}
\newcommand{\homework}[3]{
\noindent
\fbox{
\begin{minipage}{6.2in}
  {\bf CS 710: Complexity Theory} \hfill #2
  \begin{center}
    {\Large Homework #1} \\[3mm]
  \end{center}
Instructor: Dieter van Melkebeek \hfill #3
\end{minipage}
}
\bigskip

\bigskip
}

% add DRAFT to your document %
\newcommand{\draft}[0]{
\begin{center}
	{\bf \Large {\sc DRAFT} }
\end{center}
}

% example environment
\newenvironment{example}
{\smallskip \noindent \emph{Example:}}
{\hfill $\boxtimes$ \smallskip}

% some theorem environments
\newtheorem{conjecture}{Conjecture}
\newtheorem{theorem}{Theorem}
\newtheorem{proposition}{Proposition}
\newtheorem{claim}{Claim}
\newtheorem{lemma}{Lemma}
\newtheorem{corollary}{Corollary}
\newtheorem{definition}{Definition} % Use this for non-trivial
	% definitions.

% currently not used %
\newtheorem{exercise}{Exercise}
\newtheoremstyle{example}{\topsep}{\topsep}%
     {\normalfont \small}   % Body font
     {}    % Indent amount (empty = no indent, \parindent = para indent)
     {\bfseries}     % Thm head font
     {}%           Punctuation after thm head
     {\topsep}%     Space after thm head
     {}%         Thm head spec    \theoremstyle{example}
\theoremstyle{example}
%\newtheorem{example}{Example}



% Set the margins
%
\setlength{\textheight}{8.5in}
\setlength{\headheight}{.25in}
\setlength{\headsep}{.25in}
\setlength{\topmargin}{0in}
\setlength{\textwidth}{6.5in}
\setlength{\oddsidemargin}{0in}
\setlength{\evensidemargin}{0in}

% Macros
\newcommand{\myN}{\hbox{N\hspace*{-.9em}I\hspace*{.4em}}}
\newcommand{\myZ}{\hbox{Z}^+}
\newcommand{\myR}{\hbox{R}}

\newcommand{\myfunction}[3]
{${#1} : {#2} \rightarrow {#3}$ }

\newcommand{\myzrfunction}[1]
{\myfunction{#1}{{\myZ}}{{\myR}}}


% Formating Macros

\newcommand{\myheader}[4]
{\vspace*{-0.5in}
\noindent
{#1} \hfill {#3}

\noindent
{#2} \hfill {#4}

\noindent
\rule[8pt]{\textwidth}{1pt}

\vspace{1ex} 
}  % end \myheader 

%%%%%% Begin document with header and title %%%%%%%%%%%%%%%%%%%%%%%%%

\begin{document}

\pagestyle{plain}

\begin{center}
  \Large\textbf{Automated Data Configuration for NoSQL Systems}\\
  \large\textit{Kevin Lee, Akshay Mittal, Sachin Ravi}
\end{center}

\bigskip

\section{Introduction}
NoSQL storage options such as \emph{Cassandra} and \emph{MongoDB} are gaining traction in industry because traditional
RDBMS struggle with the scale associated with web services. By being distributed and decentralized, these new storage 
services allow one to horizontally scale by adding more commodity machines as necessary. Though there exist many NoSQL
options, all offering differing data models, we concentrate our efforts on column-oriented data stores, such as BigTable
and Cassandra.

Data modeling in these new systems differs from traditional RDBMS data modeling. In relational databases, where most queries
are carried out through the use of joins, a fully normalized data model is conventional; however in the NoSQL storage systems, 
we avoid using joins for scalability purposes and so it is necessary to denormalize one's data by storing repetitive information 
to optimize the system's response to certain queries.

The decision of how to denormalize one's data and to what level is dependent on the query workload one expects for his/her service.
Not only does this require the developer to predict the workload ahead of time but it may involve a complicated cost-benefit analysis that
is harder for larger, more complicated workloads.

Here, we believe it is possible to help the developer and have an automated way to do this modeling. Given a query workload and 
the normalized definition of how tables are related, it should be possible to give an optimal (or close to optimal) configuration 
for how the denormalized data should be stored. This sort of tool would help avoid extended analysis and allow one to automatically
change the data configuration if the query workload has changed enough.

\section{Related Work}
We look to previous database literature on automatically enumerating materialized views for SQL databases to solve the above problem. 
In many ways, a materialized view is similar to a table that is storing denormalized data since in both cases, data is being replicated so 
that queries can be answered more efficiently. Specifically, in \cite{agrawal2000automated}, the authors discuss an entire end-to-end solution
for automated selection of materialized views based on their relative importance to a set of queries. The main contributions of this paper are 
algorithms for: (1) producing a candidate materialized view set that is much smaller than the set of all possible materialized views for a
query workload; (2) adding to this set by merging views from the original set.

In \cite{chan99design} an algorithm is presented for selecting materialized views greedily based on their storage effectiveness.  This
storage effectiveness is defined to be the view's net benefit, dependent on query and maintenance costs, divided by its storage cost.  By
adding the notion of storage size directly into the cost model, this work is thus related to the issue of denormalized representations of
data in NoSQL systems where there is also a benefit vs storage tradeoff being examined.

In both \cite{agrawal2000automated} and \cite{chan99design}, however, the query optimizer is used extensively to evaluate the cost of materialized views for different queries.
In Cassandra, for example, we do not have access to a query optimizer so it will be necessary to build our own cost model to evaluate the cost
of a certain data configuration for a specific query. 

Nectar~\cite{gunda2010nectar} system was designed to avoid manual management of data and computation; it automates and unifies the management of
data and computation within a datacenter. Derived datasets, which are the results of computations, are uniquely identified by the programs that produce them,
and together with their programs, are automatically managed by a datacenter wide caching service. Any derived dataset can be transparently regenerated by re-executing
its program, and any computation can be transparently avoided by using previously cached results. This trade-off between storage and computation can
be leveraged to define an efficient view materialization scheme for NoSQL storage systems.

HBase~\cite{hbaseBook} is an open source, non-relational, distributed database modeled after Google's BigTable. To make best use of the features
of HBase, application developers have the onus to denormalize the schemas. If the query load changes, the initial optimal denormalization may
prove to be inefficient for the incoming queries. An automated denormalization of the schemas will incur initial processing cost but will relieve
the developer of the denormalization task.

\section{Preliminary Plan}
We plan on working with Cassandra, an open-source, column-oriented data-store created at Facebook, to evaluate our automated data configuration
system. In the first iteration, we plan to work outside the system and create a prototype that accepts a list of Cassandra queries and a normalized
view of the data, to produce an optimal denormalized data configuration. Building this system outside of Cassandra will allow us to concentrate on 
the actual project rather than spend a lot of time familiarizing ourselves with Cassandra code. We can then evaluate the automated data configuration 
produced against other manually planned options for various query workloads to see how well our system performs.

For a later goal, we would like to incorporate our system into Cassandra itself so that it can read from the query logs and automatically shift 
the existing data configuration to a more optimal one. We plan to use the ``Yahoo! Cloud Serving Benchmark" (YCSB) framework~\cite{ycsb} for the
purpose of query workloads.

\bibliographystyle{plain}
\bibliography{references}

\end{document}
