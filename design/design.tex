\documentclass[10pt]{article}

\usepackage{amsmath, amsfonts, amsthm, fullpage, amssymb, algpseudocode, algorithm, graphicx}

\setlength{\textheight}{9in}
\setlength{\textwidth}{6.5in} 
\setlength{\topmargin}{-.5in}
\setlength{\evensidemargin}{0in}
\setlength{\oddsidemargin}{0in}
\setlength{\headsep}{0.5in}
\setlength{\parskip}{0ex}
\setlength{\itemsep}{-.6ex}

\newcommand{\comment}[1]{}

\newtheorem{theorem}{Theorem}
\newtheorem{lemma}[theorem]{Lemma}
\newtheorem{invariant}[theorem]{Invariant}
\newtheorem{corollary}[theorem]{Corollary}
\newtheorem{definition}[theorem]{Definition}
\newtheorem{property}[theorem]{Property}
\newtheorem{remark}[theorem]{Remark}

\floatstyle{ruled}
\newfloat{Program}{H}{lop}[section]
\newfloat{Function}{H}{lop}[section]

\floatstyle{boxed}
\newfloat{Output}{H}{lop}[section]

\floatstyle{plain}
\newfloat{Demo}{H}{lop}[section]
\newfloat{Figure}{H}{lop}[section]
\newfloat{Myfig}{H}{lop}[section]

% These are some useful macros

\algnewcommand\algorithmicinput{\textbf{INPUT:}}
\algnewcommand\INPUT{\item[\algorithmicinput]}
\algnewcommand\algorithmicoutput{\textbf{OUTPUT:}}
\algnewcommand\OUTPUT{\item[\algorithmicoutput]}

\def\handout#1#2{
      \centerline{\framebox{\framebox{ \parbox{5in}{ \bf COS 423 \hfill
      Theory of Algorithms \hfill Spring 2013 \hfill \\
      \hfill\mbox{ } \\[1mm] \mbox{ } \hfill{\Large \bf #2 #1}\hfill
      \mbox{ }} }}}
      \vspace{0.5in}
}

\title{Precept 1}

\begin{document}

\title{\bf Automated Data Configuration for NoSQL Systems}
\author {Kevin Lee \and Akshay Mittal \and Sachin Ravi}

\date{}
\maketitle

\section{High-level Design}
\begin{enumerate}
\item Will receive SQL query workload
\item Finding Interesting Table-subsets with selection conditions imposed (based purely on table-subsets occurring in queries) 
\footnote{Might be interesting to try without selection conditions imposed and no merging}
\item Do merging to get more options (to get table-subsets that are not explicitly in any query)
\item For each of these table-subsets, we calculate score across all queries. 
Supose $Q$ is the set of queries we are considering and $m \in M$, the set of all table-subsets we are considering, is one possible table-subset,
\\
\\
$$queryCost(m, Q) = \textrm{cost to answer query using table-subset $m$}$$
$$updateCost(m,Q) = \beta \cdot \textrm{number of queries that update any table in $m$}$$
$$storageCost(m) = \alpha \cdot \textrm{size of storing $m$}$$
$$cost(m,Q) = queryCost(m,Q) + updateCost(m,Q) + storageCost(m)$$
\\
\\
Add each cost to priority queue
\item Get $k$ lowest costs out of priority queue and keep the table-subsets corresponding to these $k$ lowest costs (if having picked smaller subset, ingore larger subset if it is considered later)
\end{enumerate}

\subsection{Finding Interesting Table-Subsets}
$$ Cost(T) = \textrm{cost of all queries where subset $T$ occurs} $$

$$ WeightedCost(T) = \sum_{i} cost(Q_i) \cdot \frac{\textrm{sum of sizes of tables in T}}{\textrm{sum of sizes of all tables referenced in $Q_i$}} $$

\begin{lemma}
For $T_1 \subseteq T_2$, $Cost(T_1) \ge Cost(T_2)$.
\end{lemma}

\begin{lemma}
$WeightedCost(T) \ge C \Rightarrow Cost(T) \ge C$ $\Leftrightarrow$ $Cost(T) \not \ge C \Rightarrow WeightedCost(T) \not \ge C$ 
\end{lemma}

\begin{algorithm}[H]
  \caption{Algorithm for finding potential table-subsets in query workload}
  \begin{algorithmic}
  	\INPUT $C$ (baseline), $maxSize$ (max size for a table-subset) \\
  	$S_1 \gets \{T \, | \, T \textrm{ is a table-subset of size $1$ with } Cost(T) \ge C\}$ \\
  	$i \gets 1$ \\
  	\While{$i < maxSize$ and $|S_i| > 0$} \\
  		$i = i + 1$ \\
  		$S_i = \{\}$ \\
  		$G \gets \{T \, | \, T \textrm{ is a table subset of size $i$ and } \exists s \in S_{i-1} \textrm{such that } s \subset T\}$
  		\For{$g \in G$}
  			\If{$Cost(g) \ge C$}
  				$S_i = S_i \cup \{g\}$ 
  			\EndIf
  		\EndFor
  	\EndWhile
  \end{algorithmic}
  $S \gets S_1 \cup S_2 \cup \ldots \cup S_{maxSize}$ \\
  $R \gets \{T \, | \, T \in S \textrm{ and } Weight(T) \ge C\}$ \\
  return $R$
\end{algorithm}

When picking table-subsets, we want to pick subsets that are in expensive queries (because materializing these table-subsets will decrease a great cost).
We pass in a threshold value of the minimum total cost we would like to satisfy when considerng a certain table-subset.
We start with single table-subsets and continually work our way up to higher table-subsets, while ensuring that we still satisfy the threshold.

\subsection{Merging Table-Subsets}
The problem with only using the previous procedure is that we only consider table-subsets that exist in some query in the workload. Oftentimes, there are table-subsets
that don't occur in any query but are useful to materialize anyway.

[Insert Example]

\begin{algorithm}[H]
  \caption{Pairwise Merge}
  \begin{algorithmic}
    \INPUT $T_1$, table-subset, $T_2$ table-subset \\
    \If{$T_1 \cap T_2 = \emptyset$}
      return $\emptyset$
    \EndIf
    \\
    \\
    $newT \gets$ union of $T_1$ and $T_2$'s projection columns (only the ones on the insterseciton of $T_1$ and $T_2$'s intersecting table-subsets), intersection of $T_1$ and $T_2$'s selection conditions, and intersection of $T_1$ and $T_2$'s table-subsets
    \\
    \\
    return $newT$
  \end{algorithmic}
\end{algorithm}

\begin{algorithm}[H]
  \caption{Algorithm for producing merged table-subsets}
  \begin{algorithmic}
    \INPUT $T$, set of table-subsets \\
    $R \gets T$ \\
    \While{$|R| > 1$}
      \State $M' = \emptyset$ 
      \For{$R_1, R_2 \in R$} 
        \State $newR \gets$ Pairwise Merge of $R_1$ and $R_2$
        \State $M' \gets M' \cup newR$
      \EndFor
      \\
      \If{$M' = \emptyset$}
        \State return $R - T$ 
      \EndIf
      \\
      \ForAll{$m \in M'$}
        \State Remove both parents of $m$ from R
      \EndFor   
      \State $R \gets R \cup M'$
      \\
    \EndWhile 
    \State return $R - T$
  \end{algorithmic}
\end{algorithm}

\subsection{Pruning}

\subsection{Cost Model}
The basic thing we need to write is $cost(q, T)$ where $q$ is a normalized query and $T = \{T_1, \ldots, T_k\}$ is a subset of tables with possible selection conditions imposed (the possible denormalized option). 
\\
$cost(q,T)$
\\
\\


\bibliographystyle{plain}
\bibliography{references}

\end{document}