\section{Introduction}
\label{sec:intro}

NewSQL storage options such as MemSQL and SQLFire are gaining traction in industry because traditional RDBMS struggle with the scale associated with web services.  These NewSQL distributed storage services allow one to horizontally scale by adding more commodity machines as necessary without giving up strong transactional and consistency requirements as many NoSQL storage services do.  Though there exist many NewSQL options, all offering different internal architectures, we concentrate our efforts on in-memory databases that allow one to configure the placement of data across data servers through partitioning and replication, such as VMWare's vFabric SQLFire.

Depending on the placement of data in such NewSQL systems, the time required to execute a given query workload can vary greatly.  When considering the optimal data placement to handle a given query workload, the main nontrivial goal is to colocate related data necessary for the execution of important joins onto each data server.  Each data server can then execute an assigned portion of such important joins through only having to use the colocated data stored in its own memory.  However, if the data placement is configured poorly, many important joins may require gathering missing data from other data servers.  Such distributed joins can be very expensive to perform and are generally not supported.

The optimality of a data placement configuration for a NewSQL system is thus highly dependent on the query workload to be handled.  This presents key technical challenges as a developer must be able to analyze a query workload to understand which joins are important and do a complicated cost-benefit analysis to determine how to place the data.  We present an automated method for doing such an analysis, when given a data model and a query workload, to produce an optimal data placement configuration onto a set of servers.  We evaluate our method using an Emulab environment using SQLFire and the TPC-H query workload and find our method is able to adapt to varying query workloads to produce highly optimized data placement configuration when compared to other strategies.